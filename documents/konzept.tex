\documentclass[paper=a4,fontsize=12pt]{scrartcl}
\usepackage{geometry}
\geometry{verbose, a4paper, tmargin=25mm, bmargin=25mm, lmargin=25mm, rmargin=25mm}

\usepackage[latin1]{inputenc}
\usepackage[ngerman]{babel}

\usepackage{fancyhdr} %Paket laden
\pagestyle{fancy} %eigener Seitenstil
\fancyhf{} %alle Kopf- und Fu�zeilenfelder bereinigen

\fancyhead[L]{Bildverschl�sselung mit Matlab} 
\fancyhead[C]{Konzept} %zentrierte Kopfzeile
\fancyhead[R]{Marco Berger, Andy Pollari} %Kopfzeile rechts
\renewcommand{\headrulewidth}{0.4pt} %obere Trennlinie
\fancyfoot[C]{\thepage} %Seitennummer
\renewcommand{\footrulewidth}{0.4pt} %untere Trennlinie
 
\begin{document}
\section*{Bildverschl�sselung mit Matlab}
\subsection*{Konzept}
In der heutigen Zeit werden kryptographische Verschl�sselungen in der Informatik immer wichtiger.
Dies kommt von dem immer gr�sseren Verlangen nach Privatsph�re und Datenschutz in der digitalen Welt.
In dieser Arbeit befassen wir uns mit der Anwendung verschiedener Verschl�sselungsalgorithmus
angewandt auf Bilder implementiert in Matlab. \\
Es ist zu erw�hnen, dass es grunds�tzlich zwei verschiedene Verschl�sselungsverfahren gibt:
\begin{itemize}
  \item Die symmetrische Verschl�sselung
  \item Die asymmetrische Verschl�sselung 
\end{itemize}
Bei der symmetrisch Verschl�sselung wird mit einem Schl�ssel ver- wie auch entschl�sselt.
Bei der asymmetrisch hingegen gibt es zwei Schl�ssel: Einen �ffentlichen Schl�ssel zum verschl�sseln
und einen privaten Schl�ssel zum entschl�sseln. \\ \\
Es gibt verschiedene asymetrische Verschl�sselungsverfahren wie RSA, Merkle-Hellmann, RSA, \ldots \\
Auch bei den symetrischen Verschl�sselungsverfahren gibt es verschiedene wie DES, AES, One-Time-Pad, \ldots \\
Im Rahmen dieser Arbeit konzentrieren wir uns bei der symmetrische Verschl�sselung auf das \textit{One-Time-Pad} 
und bei den asymmetrisch Verschl�sselungsverfahren auf RSA. \\ \\
In dieser Arbeit untersuchen wir einerseits die Performance verschiedener \\Verschl�sslungsalgorithmen
und ob man aus den verschl�sselten Bilddaten irgendwelche R�ckschl�sse ziehen kann.
Folgende Fragestellungen haben wir uns f�r diese Arbeit gestellt:
\begin{itemize}
  \item Wie unterschiedlich ist ein verschl�sseltes Bild mit dem Ausgangsbild?
  \item Kann man einen popul�res Bildformat wie JPEG oder PNG asymetrische verschl�sseln, so dass
  das Bild immer noch korrekt interpretiert werden kann?
  \item Wie gut eignet sich Matlab f�r g�ngige Verschl�sselungsalgorithmen
\end{itemize}
Die Verschl�sselungen wenden wir auf JPEG Daten an.

\section*{Erwartete Resultate}
Uns ist bewusst, dass wir mit Matlab f�r eine asymmetrisch Verschl�sselung keine
grossen Primzahlen nutzen k�nnen. Trotz relativ kleiner Primzahl, erwarten wir
bei der asymmetrischen Verschl�sselung im Vergleich zu der symmetrischen Ver- und Entschl�sselung
grosse Performanceeinbussen. \\
Wir erwarten, dass wir bei einem Bildformat wie JPEG nur Darstellungsrelevanten
Daten ver- resp. entschl�sseln k�nnen.
Nach der Verschl�sselung kann das Bild auf keine Weise 
(ausser den nicht verschl�sselten Metadaten) mit dem Originalbild in Verbindung gebracht werden


\section*{Zeitplan}
\begin{tabular}{rl}
	03.12.2014	&	Abschluss und Abnahme des Konzepts durch M. Stampfli \\
	17.12.2014	&	Erfolgreiches Einlesen eines Bildes mit Datenmanipulation \\
	24.12.2014	&	Symmetrische und Asymmetrische Verschl�sselung angewandt auf Bilder \\
	07.01.2015	&	Abschluss der Implementierungen.  \\
				&	Festhalten der Ergebnisse  \\
	16.01.2015	&	Dokumentation fertiggestellt.  \\
				&	Abgabe der Arbeit  \\
				&	Pr�sentation  \\
	
\end{tabular}

\end{document}
