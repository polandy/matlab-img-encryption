% Die Titelseite
% Im folgenden kommen ein paar Variablen, die auszuf�llen sind
% Bisher steht dort nur Musterinhalt
% Au�erdem m�ssen zei Dateien erstellt werden, Bild/Logo/Emblem des Fachgebietes
% sowie der Universit�t

\newcommand{\trtitle}{Image encryption with matlab}
\newcommand{\trtype}{Modul: Programmierem mit Matlab/Octave}
\newcommand{\trauthor}{Marco Berger\\Andy Pollari}
% \newcommand{\trstrasse}{Musterstra�e 666}
% \newcommand{\trmatrikelnummer}{0815}
\newcommand{\trort}{Bern}
\newcommand{\trbetreuer}{Prof. Marx Stampfli}
% \newcommand{\trprof}{Prof. Dr. Udo Seltsam}
% \newcommand{\trfachgebiet}{Kryptografie}
% \newcommand{\trinstitut}{gute Wissenschaft}
% \newcommand{\trfakultaet}{IV Wissenschaft}
% \newcommand{\truni}{Universit�t Musterstadt}
\newcommand{\trdate}{\today}

\thispagestyle{empty}

% Kopfzeile mit Logos.
% Eventuell die \hspace{} je nach Logogr��e anpassen
\begin{tabular}{lcr}
  \includegraphics[scale=0.8]{logo_bfh_de.jpg}% & % dein_unilogo.jpg/.eps im Verzeichnis "bilder" ablegen
 % \hspace{2cm} \truni \hspace{2cm} &
 % \includegraphics[scale=0.8]{dein_fglogo} % dein_fglogo.jpg/.eps im Verzeichnis "bilder" ablegen, Fachgebietslogo
  \\
\end{tabular}

\rule{\textwidth}{0.4pt}

\vspace{2.5cm}
\begin{center}
  \textbf{\LARGE \trtitle}
\end{center}
\vspace{2cm}

% \begin{center}
%   \textbf{\trtype} \\
%   am Fachgebiet \trfachgebiet \\
%   \trprof \\
%   Institut f�r \trinstitut \\
%   Fakult�t \trfakultaet \\
%   \truni \\[0.5cm]
%   vorgelegt von \\
%   \textbf{\trauthor}
% \end{center}

\vspace{1cm}


\begin{center}
\begin{tabular}{ll}
Betreuer: & \trbetreuer \\
\end{tabular}
\end{center}

\vfill

\begin{tabular}{l}
% \trauthor \\
% Matrikelnummer:  \trmatrikelnummer \\
% \trstrasse \\
% \trort
\end{tabular}

\rule{\textwidth}{0.4pt}
