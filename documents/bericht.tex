\documentclass[paper=a4,fontsize=12pt]{scrartcl}
\usepackage{geometry}
\usepackage{graphicx}
\geometry{verbose, a4paper, tmargin=25mm, bmargin=25mm, lmargin=25mm, rmargin=25mm}

\usepackage[utf8]{inputenc}
\usepackage[ngerman]{babel}


\usepackage{amsmath}

% Grafiken einbinden 
\usepackage{graphicx}
\usepackage{here} % http://overspice.blogspot.ch/2007/10/latex-bilder-richtig-plazieren.html

\usepackage{fancyhdr} %Paket laden
\pagestyle{fancy} %eigener Seitenstil
\fancyhf{} %alle Kopf- und Fu�zeilenfelder bereinigen
\fancyhead[L]{\includegraphics[width=3cm]{img/logo_bfh_de.jpg}} %Kopfzeile links
\fancyhead[C]{} %zentrierte Kopfzeile
\fancyhead[R]{Marco Berger, Andy Pollari} %Kopfzeile rechts
\renewcommand{\headrulewidth}{0.4pt} %obere Trennlinie
\fancyfoot[C]{\thepage} %Seitennummer
\renewcommand{\footrulewidth}{0.4pt} %untere Trennlinie

\makeatletter
\let\ps@plain\ps@fancy 
\makeatother


\usepackage{pgfplots}
\usepackage{filecontents}

\begin{filecontents*}{data.csv}
imgNames,width,height,megapixel,time
large-1.jpg,5184,3456,17.915904,0.0226542369937176
large-10.jpg,5184,3456,17.915904,0.02146350620254
large-2.jpg,5074,3383,17.165342,0.0214863869118607
large-4.jpg,5184,3456,17.915904,0.0218580817000087
large-5.jpg,5184,3456,17.915904,0.0217852369927837
large-6.jpg,5184,3456,17.915904,0.0220588715981288
large-7.jpg,11029,4695,51.781155,0.0621215927588625
large-8.jpg,5530,4142,22.90526,0.0278047313571249
large-9.jpg,3456,5184,17.915904,0.0213010064710382
medium-1.jpg,3000,2000,6,0.00726112305992598
medium-10.jpg,3000,2000,6,0.00749226491938986
medium-2.jpg,3000,2000,6,0.00722563461281637
medium-3.jpg,4085,2000,8.17,0.0102907157552829
medium-4.jpg,3000,2000,6,0.00744230092148555
medium-5.jpg,3000,2000,6,0.00724011016361108
medium-6.jpg,3000,2000,6,0.00741755239915911
medium-7.jpg,4698,2000,9.396,0.0112199527256528
medium-8.jpg,2670,2000,5.34,0.00675494573536265
medium-9.jpg,1333,2000,2.666,0.0035222283756284
small-1.jpg,1536,1024,1.572864,0.00216059269442299
small-10.jpg,1536,1024,1.572864,0.00209195056646099
small-2.jpg,1536,1024,1.572864,0.00207794196891772
small-3.jpg,2091,1024,2.141184,0.00275035465099447
small-4.jpg,1536,1024,1.572864,0.00213724503185088
small-5.jpg,1536,1024,1.572864,0.00212697206031915
small-6.jpg,1536,1024,1.572864,0.00201490327997303
small-7.jpg,2405,1024,2.46272,0.00306788286197516
small-8.jpg,1367,1024,1.399808,0.00195560021703987
small-9.jpg,683,1024,0.699392,0.00102636324666992

\end{filecontents*}


\begin{document}
\title{Bildverschlüsselung mit Matlab}
\author{Marco Berger, Andy Pollari}
\date{14.01.2015}
\maketitle 

\section{Einleitung}
In dieser Arbeit befassen wir uns mit der Anwendung verschiedener Verschlüsselungsalgorithmus
angewandt auf Bilder implementiert in Matlab. \\
Es ist zu erwähnen, dass es grundsätzlich zwei verschiedene Verschlüsselungsverfahren gibt:
\begin{itemize}
  \item Die symmetrische Verschlüsselung 
  \item Die asymmetrische Verschlüsselung 
\end{itemize}
Bei der symmetrisch Verschlüsselung wird mit einem Schlüssel ver- wie auch entschlüsselt.
Bei der asymmetrisch hingegen gibt es zwei verschiedene Schlüssel: Einen öffentlichen Schlüssel zum verschlüsseln
und einen privaten Schlüssel zum entschlüsseln. \\ \\
Es gibt verschiedene asymetrische Verschlüsselungsverfahren wie RSA, Merkle-Hellmann, RSA, \ldots \\
Auch bei den symetrischen Verschlüsselungsverfahren gibt es verschiedene wie DES, AES, One-Time-Pad, \ldots \\
Im Rahmen dieser Arbeit konzentrieren wir uns bei der symmetrische Verschlüsselung auf das \textit{One-Time-Pad} 
und bei den asymmetrisch Verschlüsselungsverfahren auf RSA. \\ \\
In dieser Arbeit haben wir festgestellt, dass sich Matlab nur bedingt eignet, um Bilder zu Verschlüsseln.
Für die symmetrische Verschlüsselung stiessen wir auf keine grösseren Probleme. Bei der asymmetrischen Verschlüsselung
trafen wir auf ein grösseres Problem bezüglich Primzahlen. Dieses Problem erläutern wir später im Kapitel \ref{results} \textit{Ergebnisse, Resultate}

\newpage 
\section{Grundlagen} 
In dieser Arbeit beschränken wir uns auf folgende Verschlüsselungsverfahren:
\begin{itemize}
  \item One Time Pad, als symmetrische Verschlüsselung
  \item Rivest, Shamir und Adleman (RSA), als asymmetrische Verschlüsselung.
\end{itemize} 
Daher beschränken wir uns auschliesslicht auf diese beiden Verfahren. 

\subsection{One-Time-Pad}
Beim One-Time-Pad haben wir einen Keystream der aus random Bits besteht.
Dieser Keystream muss mindestens so viele Bits lang sein, wie die zu verschlüsselnde Nachricht selbst.
In unserem Projekt ist die zu verschlüsselnde Nachricht ein JPG-Bild. \\
Die Idee beim One-Time-Pad ist, dass der Keystream nur einmal zum ver- respektive entschlüsseln verwendet wird. \\ \\
Die Verschlüsselung beim One-Time-Pad wird realisiert, 
indem das Bit an der Position $i$ des Bildes $m$ mit dem $i$-ten Bit des Keystreams $r$ XOR verknüpft wird.
% \begin{itemize}
%   \item Plaintext: $m = m_1 || m_2 || \cdots || m_k$
%   \item Ciphertext: $c = c_1 || c_2 | \cdots || c_k$
%   \item Keystream: $r = r_1 || r_2 | \cdots || r_k$
% \end{itemize}

\begin{figure}[H] 
	\centering
	\makebox[\textwidth]{\includegraphics[width=7cm]{img/one-time-pad_enc.jpg}}
	\caption[One Time Pad Encryption]{One-Time-Pad Encryption}  
	\label{One-Time-Pad Encryption}  
\end{figure}
\begin{figure}[H] 
	\centering
	\makebox[\textwidth]{\includegraphics[width=7cm]{img/one-time-pad_dec.jpg}}
	\caption[One Time Pad Decryption]{One-Time-Pad Decryption}  
	\label{One-Time-Pad-enc} 
\end{figure}


\subsection{Rivest, Shamir und Adleman (RSA)} \label{RSA-intro}
RSA ist ein asymmetrisches kryptographisches Verfahren. Es wird zur Verschlüsselung aber auch zur digitalen Signatur verwendet.
In dieser Arbeit befassen wir uns ausschliesslich mit dem Verschlüsselungsverfahren von RSA.
\\ \\
RSA verwendet ein Schlüsselpaar bestehend aus einen privaten Schlüssel (private key) und einem öffentlichen Schlüssel  (public key).
Den private key verwendet man dabei um Daten zu entschlüsseln, die mit dem public key verschlüsselt worden sind. 

\subsubsection{Schlüssel Generierung}
Der public key ist ein Zahlenpaar $(e,N)$, und der private key ist auch ein Zahlenpaar $(d,N)$.
$e$ des public keys wird auch den Verschlüsselungsexponenten genannt.
$d$ des private keys wird auch den Entschlüsselungsexponenten genannt. 
Bei beiden Zahlenpaaren ist N gleich und wird \textit{RSA Modul} genannt.
Eine kurze grobe Beschreibung, wie die oben genannten Schlüssel generiert werden:
\begin{itemize}
  \item Die Primzahlen $p$ und $q$ zufällig wählen, $p \neq q$
  \item Den RSA-Modul $N$ berechnen \begin{align} N = p \cdot q \end{align}
  \item $\varphi(N)$ berechnen \begin{align} \varphi(N) = (p-1) \cdot (q-1)\end{align}
  \item $e$ wählen, welches teilerfremd von $\varphi(N)$ ist
  \item $d$ wählen, wobei gilt \begin{align} e \cdot d \equiv_N mod \varphi(N)\end{align} $d$ ist also das multiplikativ Inverse Element von $e$ im Bezug auf $\varphi(N)$
\end{itemize}

\subsubsection{Verschlüsselung / Entschlüsselung}
Möchten wir eine Nachricht $m$ verschlüsseln, so wird die Nachricht $m$ mit dem Verschlüsselungsexponenten $e$ potenziert.
Man erhält so den Ciphertext $c$ (Geheimtext).
\begin{align}
	c \equiv_N m^e
\end{align}

Um den Ciphertext $c$ zu entschlüsseln, muss $c$ mit dem Entschlüsselungsexponenten $d$ potenziert werden.
So erhalten wir wieder die Ursprungsnachricht $m$.
\begin{align}
	m \equiv_N c^d
\end{align}

 

\newpage
\section{Vorgehen, Methoden Analysen}
Da wir die Grundlagen der zu verwendenden Verschlüsselungen bereits durch unsere Studium-Vertiefung IT-Security hatten,
konnten wir uns schnell der Implementierung widmen.

\subsection{Implementierung der symmetrischen Verschlüsselung}
Wie bereits erwähnt haben wir uns für eine One Time Pad Verschlüsselung entschieden.
Dabei wird ein Bild in eine Matrix $M_M$ gelesen. 
Für das One Time Pad erstellen darauf hin eine Matrix $M_R$ mit Zufallswerten und exakt der Grösse, des eingelesenen Bildes.
Die beiden Matrizen werden XOR miteinander verknüpft, was eine neue Matrix $M_C$ ergibt und dem verschlüsselten Bild entspricht.
Nachträglich haben wir die Zeitmessung eingebaut, um Auswertungen machen zu können.


\subsection{Implementierung der asymmetrischen Verschlüsselung}
Bei der asymmetrischen Verschlüsselung haben wir uns für die RSA Verschlüsselung entschieden.
Die Implementiertung erfolgte erst ziemlich genau so, wie im Grundlagen Kapitel \ref{RSA-intro} beschrieben:
Erst die Schlüsselgenerierung und anschliessend wurde jeder Wert des Bildes mit den Bilder ver- und entschlüsselt.
Bei der RSA Implementierung stiessen wir auf einige Probleme, die wir im Kapitel \ref{problems-RSA} erläutern.
Nach der vollständigen Implementierung der Ver- resp. Entschlüsselung, wurden auch hier noch Zeitmessungen eingebaut, um Auswertungen
machen zu können.

\newpage
\section{Ergebnisse, Resultate} \label{results} 

\subsection{Symmetrische Verschlüsselung}
\subsubsection{Performance bei der symmetrischen Verschlüsselung}
\begin{tikzpicture}
\begin{axis}[width=\textwidth, height=0.6\textwidth,
		title={Performance - Symmetrische Verschlüsselung},
		ylabel={Megapixel},
		xlabel={Sekunden}
		]
\addplot table [x=time, y=megapixel, col sep=comma] {data.csv};
\end{axis}
\end{tikzpicture} \\
Unserer Meinung ist die Performance bei der symmetrischer Verschlüsselung sehr akzeptabel.
Die obere Grafik zeigt, dass die Verschlüsselungsdauer linear zur Anzahl Bildpunkten ist.
Um ein Bild vom 50 Megapixel zu Verschlüsseln braucht unser Matlab Programm 0.06s.
Die Verschlüsselungsdauer ist mit der Entschlüsselungsdauer identisch, da die gleichen Operationen (XOR) durchgeführt werden.
Bezüglich der Performance eignet sich Matlab unserer Meinung sehr gut für eine symmetrische Verschlüsselung mit dem One-Time-Pad.

\subsubsection{Wiedererkennbarkeit des Bildes}
Mehrere Versuche zeigten, dass das Originalbild nach der Verschlüsselung in keiner Weise wiedererkennbar ist.
Auch in diesem Punkt, überzeugt die symmetrische Verschlüsselung mit dem One-Time-Pad.
Die Abbildung \ref{sym-enc-result} zeigt ein Beispiel des Resultates.
\begin{figure}[H] 
	\centering
	\makebox[\textwidth]{\includegraphics[width=\textwidth]{img/sym-enc-result.jpg}}
	\caption[One Time Pad - Ver- und Entschlüsselung]{One Time Pad - Ver- und Entschlüsselung}  
	\label{sym-enc-result} 
\end{figure}

\subsection{Asymmetrische Verschlüsselung}
\subsubsection{Performance bei der asymmetrischen Verschlüsselung}
\subsubsection{Wiedererkennbarkeit des Bildes}
 
 \subsection{Eignet sich nun Matlab für Bildverschlüsselung?} \label{problems-RSA}
 Probleme Notiz
 \begin{itemize}
  \item wenn jedes Pixel mit dem gleichen Exponenten e gerechnet wird, so wird das Bild nicht wirklich verschlüsselt.
\end{itemize}
\newpage
\section{Ahnang}

\newpage
\section{Literatur}
\end{document}