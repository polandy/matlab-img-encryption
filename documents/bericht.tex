\documentclass[paper=a4,fontsize=12pt]{scrartcl}
\usepackage{geometry}
\usepackage{graphicx}
\geometry{verbose, a4paper, tmargin=25mm, bmargin=25mm, lmargin=25mm, rmargin=25mm}

\usepackage[utf8]{inputenc}
\usepackage[ngerman]{babel}

\usepackage{amsmath}

\usepackage{listings}

% Grafiken einbinden 
\usepackage{graphicx}
\usepackage{here} % http://overspice.blogspot.ch/2007/10/latex-bilder-richtig-plazieren.html

\usepackage{fancyhdr} %Paket laden
\pagestyle{fancy} %eigener Seitenstil
\fancyhf{} %alle Kopf- und Fu�zeilenfelder bereinigen
\fancyhead[L]{\includegraphics[width=3cm]{img/logo_bfh_de.jpg}} %Kopfzeile links
\fancyhead[C]{} %zentrierte Kopfzeile
\fancyhead[R]{Marco Berger, Andy Pollari} %Kopfzeile rechts
\renewcommand{\headrulewidth}{0.4pt} %obere Trennlinie
\fancyfoot[C]{\thepage} %Seitennummer
\renewcommand{\footrulewidth}{0.4pt} %untere Trennlinie

\makeatletter
\let\ps@plain\ps@fancy 
\makeatother


\usepackage{pgfplots}
\usepackage{filecontents}

\begin{filecontents*}{data.csv}
imgNames,width,height,megapixel,time
large-1.jpg,5184,3456,17.915904,0.0226542369937176
large-10.jpg,5184,3456,17.915904,0.02146350620254
large-2.jpg,5074,3383,17.165342,0.0214863869118607
large-4.jpg,5184,3456,17.915904,0.0218580817000087
large-5.jpg,5184,3456,17.915904,0.0217852369927837
large-6.jpg,5184,3456,17.915904,0.0220588715981288
large-7.jpg,11029,4695,51.781155,0.0621215927588625
large-8.jpg,5530,4142,22.90526,0.0278047313571249
large-9.jpg,3456,5184,17.915904,0.0213010064710382
medium-1.jpg,3000,2000,6,0.00726112305992598
medium-10.jpg,3000,2000,6,0.00749226491938986
medium-2.jpg,3000,2000,6,0.00722563461281637
medium-3.jpg,4085,2000,8.17,0.0102907157552829
medium-4.jpg,3000,2000,6,0.00744230092148555
medium-5.jpg,3000,2000,6,0.00724011016361108
medium-6.jpg,3000,2000,6,0.00741755239915911
medium-7.jpg,4698,2000,9.396,0.0112199527256528
medium-8.jpg,2670,2000,5.34,0.00675494573536265
medium-9.jpg,1333,2000,2.666,0.0035222283756284
small-1.jpg,1536,1024,1.572864,0.00216059269442299
small-10.jpg,1536,1024,1.572864,0.00209195056646099
small-2.jpg,1536,1024,1.572864,0.00207794196891772
small-3.jpg,2091,1024,2.141184,0.00275035465099447
small-4.jpg,1536,1024,1.572864,0.00213724503185088
small-5.jpg,1536,1024,1.572864,0.00212697206031915
small-6.jpg,1536,1024,1.572864,0.00201490327997303
small-7.jpg,2405,1024,2.46272,0.00306788286197516
small-8.jpg,1367,1024,1.399808,0.00195560021703987
small-9.jpg,683,1024,0.699392,0.00102636324666992
\end{filecontents*}

\begin{filecontents*}{performance-asymmetric-enc.csv}
imgName,width,height,megapixel,time
a.jpg,1536,1024,1.572864,0.750267797689702
large-1.jpg,5184,3456,17.915904,48.3300696414079
large-10.jpg,5184,3456,17.915904,115.412466811298
large-2.jpg,5074,3383,17.165342,79.4360124620484
large-4.jpg,5184,3456,17.915904,205.940519027878
large-5.jpg,5184,3456,17.915904,189.149459128049
large-6.jpg,5184,3456,17.915904,18.6510654472338
large-7.jpg,11029,4695,51.781155,28.0440322907512
large-8.jpg,5530,4142,22.90526,80.6949688588877
large-9.jpg,3456,5184,17.915904,76.0953233697962
medium-1.jpg,3000,2000,6,54.9303422487161
medium-10.jpg,3000,2000,6,47.02039465021
medium-2.jpg,3000,2000,6,59.1426434783908
medium-3.jpg,4085,2000,8.17,3.81727045278589
medium-4.jpg,3000,2000,6,38.8183215645549
medium-5.jpg,3000,2000,6,34.1271336261442
medium-6.jpg,3000,2000,6,26.0031729473436
medium-7.jpg,4698,2000,9.396,36.4283156716049
medium-8.jpg,2670,2000,5.34,19.6319595879978
medium-9.jpg,1333,2000,2.666,29.3038758987683
small-1.jpg,1536,1024,1.572864,2.50203404836328
small-10.jpg,1536,1024,1.572864,13.2850189256153
small-2.jpg,1536,1024,1.572864,0.3256433915375
small-3.jpg,2091,1024,2.141184,22.4676966410185
small-4.jpg,1536,1024,1.572864,12.8615002647625
small-5.jpg,1536,1024,1.572864,5.75673603412868
small-6.jpg,1536,1024,1.572864,11.8026622872678
small-7.jpg,2405,1024,2.46272,23.761862246923
small-8.jpg,1367,1024,1.399808,13.0273279720874
small-9.jpg,683,1024,0.699392,1.24491044303591
\end{filecontents*}


\begin{document}
\title{Bildverschlüsselung mit Matlab}
\author{Marco Berger, Andy Pollari}
\date{14.01.2015}
\maketitle 

\section{Einleitung}
In dieser Arbeit befassen wir uns mit der Anwendung verschiedener Verschlüsselungsalgorithmus
angewandt auf Bilder implementiert in Matlab. \\
Es ist zu erwähnen, dass es grundsätzlich zwei verschiedene Verschlüsselungsverfahren gibt:
\begin{itemize}
  \item Die symmetrische Verschlüsselung 
  \item Die asymmetrische Verschlüsselung 
\end{itemize}
Bei der symmetrisch Verschlüsselung wird mit einem Schlüssel ver- wie auch entschlüsselt.
Bei der asymmetrisch hingegen gibt es zwei verschiedene Schlüssel: Einen öffentlichen Schlüssel zum verschlüsseln
und einen privaten Schlüssel zum entschlüsseln. \\ \\
Es gibt verschiedene asymetrische Verschlüsselungsverfahren wie RSA, Merkle-Hellmann, RSA, \ldots \\
Auch bei den symetrischen Verschlüsselungsverfahren gibt es verschiedene wie DES, AES, One-Time-Pad, \ldots \\
Im Rahmen dieser Arbeit konzentrieren wir uns bei der symmetrische Verschlüsselung auf das \textit{One-Time-Pad} 
und bei den asymmetrisch Verschlüsselungsverfahren auf RSA. \\ \\
In dieser Arbeit haben wir festgestellt, dass sich Matlab nur bedingt eignet, um Bilder zu Verschlüsseln.
Für die symmetrische Verschlüsselung stiessen wir auf keine grösseren Probleme. Bei der asymmetrischen Verschlüsselung
trafen wir auf ein grösseres Problem bezüglich Primzahlen. Dieses Problem erläutern wir später im Kapitel \ref{results} \textit{Ergebnisse, Resultate}

\newpage 
\section{Grundlagen} 
In dieser Arbeit beschränken wir uns auf folgende Verschlüsselungsverfahren:
\begin{itemize}
  \item One Time Pad, als symmetrische Verschlüsselung
  \item Rivest, Shamir und Adleman (RSA), als asymmetrische Verschlüsselung.
\end{itemize} 
Daher beschränken wir uns auschliesslicht auf diese beiden Verfahren. 

\subsection{One-Time-Pad}
Beim One-Time-Pad haben wir einen Keystream der aus random Bits besteht.
Dieser Keystream muss mindestens so viele Bits lang sein, wie die zu verschlüsselnde Nachricht selbst.
In unserem Projekt ist die zu verschlüsselnde Nachricht ein JPG-Bild. \\
Die Idee beim One-Time-Pad ist, dass der Keystream nur einmal zum ver- respektive entschlüsseln verwendet wird. \\ \\
Die Verschlüsselung beim One-Time-Pad wird realisiert, 
indem das Bit an der Position $i$ des Bildes $m$ mit dem $i$-ten Bit des Keystreams $r$ XOR verknüpft wird.
% \begin{itemize}
%   \item Plaintext: $m = m_1 || m_2 || \cdots || m_k$
%   \item Ciphertext: $c = c_1 || c_2 | \cdots || c_k$
%   \item Keystream: $r = r_1 || r_2 | \cdots || r_k$
% \end{itemize}

\begin{figure}[H] 
	\centering
	\makebox[\textwidth]{\includegraphics[width=7cm]{img/one-time-pad_enc.jpg}}
	\caption[One Time Pad Encryption]{One-Time-Pad Encryption}  
	\label{One-Time-Pad Encryption}  
\end{figure}
\begin{figure}[H] 
	\centering
	\makebox[\textwidth]{\includegraphics[width=7cm]{img/one-time-pad_dec.jpg}}
	\caption[One Time Pad Decryption]{One-Time-Pad Decryption}  
	\label{One-Time-Pad-enc} 
\end{figure}


\subsection{Rivest, Shamir und Adleman (RSA)} \label{RSA-intro}
RSA ist ein asymmetrisches kryptographisches Verfahren. Es wird zur Verschlüsselung aber auch zur digitalen Signatur verwendet.
In dieser Arbeit befassen wir uns ausschliesslich mit dem Verschlüsselungsverfahren von RSA.
\\ \\
RSA verwendet ein Schlüsselpaar bestehend aus einen privaten Schlüssel (private key) und einem öffentlichen Schlüssel  (public key).
Den private key verwendet man dabei um Daten zu entschlüsseln, die mit dem public key verschlüsselt worden sind. 

\subsubsection{Schlüssel Generierung}
Der public key ist ein Zahlenpaar $(e,N)$, und der private key ist auch ein Zahlenpaar $(d,N)$.
$e$ des public keys wird auch den Verschlüsselungsexponenten genannt.
$d$ des private keys wird auch den Entschlüsselungsexponenten genannt. 
Bei beiden Zahlenpaaren ist N gleich und wird \textit{RSA Modul} genannt.
Eine kurze grobe Beschreibung, wie die oben genannten Schlüssel generiert werden:
\begin{itemize}
  \item Die Primzahlen $p$ und $q$ zufällig wählen, $p \neq q$
  \item Den RSA-Modul $N$ berechnen \begin{align} N = p \cdot q \end{align}
  \item $\varphi(N)$ berechnen \begin{align} \varphi(N) = (p-1) \cdot (q-1)\end{align}
  \item $e$ wählen, welches teilerfremd von $\varphi(N)$ ist
  \item $d$ wählen, wobei gilt \begin{align} e \cdot d \equiv_N mod \varphi(N)\end{align} $d$ ist also das multiplikativ Inverse Element von $e$ im Bezug auf $\varphi(N)$
\end{itemize}

\subsubsection{Verschlüsselung / Entschlüsselung}
Möchten wir eine Nachricht $m$ verschlüsseln, so wird die Nachricht $m$ mit dem Verschlüsselungsexponenten $e$ potenziert.
Man erhält so den Ciphertext $c$ (Geheimtext).
\begin{align}
	c \equiv_N m^e
\end{align}

Um den Ciphertext $c$ zu entschlüsseln, muss $c$ mit dem Entschlüsselungsexponenten $d$ potenziert werden.
So erhalten wir wieder die Ursprungsnachricht $m$.
\begin{align}
	m \equiv_N c^d
\end{align}

 

\newpage
\section{Vorgehen, Methoden Analysen}
Da wir die Grundlagen der zu verwendenden Verschlüsselungen bereits durch unsere Studium-Vertiefung IT-Security hatten,
konnten wir uns schnell der Implementierung widmen.

\subsection{Implementierung der symmetrischen Verschlüsselung}
Wie bereits erwähnt haben wir uns für eine One Time Pad Verschlüsselung entschieden.
Dabei wird ein Bild in eine Matrix $M_M$ gelesen. 
Für das One Time Pad erstellen darauf hin eine Matrix $M_R$ mit Zufallswerten und exakt der Grösse, des eingelesenen Bildes.
Die beiden Matrizen werden XOR miteinander verknüpft, was eine neue Matrix $M_C$ ergibt und dem verschlüsselten Bild entspricht.
Nachträglich haben wir die Zeitmessung eingebaut, um Auswertungen machen zu können.


\subsection{Implementierung der asymmetrischen Verschlüsselung}
Bei der asymmetrischen Verschlüsselung haben wir uns für die RSA Verschlüsselung entschieden.
Die Implementiertung erfolgte erst ziemlich genau so, wie im Grundlagen Kapitel \ref{RSA-intro} beschrieben:
Erst die Schlüsselgenerierung und anschliessend wurde jeder Wert des Bildes mit den Bilder ver- und entschlüsselt.
Bei der RSA Implementierung stiessen wir auf einige Probleme, die wir im Kapitel \ref{problems-RSA} erläutern.
Nach der vollständigen Implementierung der Ver- resp. Entschlüsselung, wurden auch hier noch Zeitmessungen eingebaut, um Auswertungen
machen zu können.

\newpage
\section{Ergebnisse, Resultate} \label{results} 

\subsection{Symmetrische Verschlüsselung}
\subsubsection{Performance bei der symmetrischen Verschlüsselung}
\begin{tikzpicture}
\begin{axis}[width=\textwidth, height=0.6\textwidth,
		title={Performance - Symmetrische Verschlüsselung},
		ylabel={Megapixel},
		xlabel={Sekunden}
		]
\addplot table [x=time, y=megapixel, col sep=comma] {data.csv};
\end{axis}
\end{tikzpicture} \\
Unserer Meinung ist die Performance bei der symmetrischer Verschlüsselung sehr akzeptabel.
Die obere Grafik zeigt, dass die Verschlüsselungsdauer wie erwartet linear zur Anzahl Bildpunkten ist.
Um ein Bild von 50 Megapixel zu Verschlüsseln braucht unser Matlab Programm 0.06s.
Die Verschlüsselungsdauer ist mit der Entschlüsselungsdauer identisch, da die gleichen Operationen (XOR) durchgeführt werden.
Bezüglich der Performance eignet sich Matlab unserer Meinung sehr gut für eine symmetrische Verschlüsselung mit dem One-Time-Pad.

\subsubsection{Wiedererkennbarkeit des Bildes}
Mehrere Versuche zeigten, dass das Originalbild nach der Verschlüsselung in keiner Weise wiedererkennbar ist.
Auch in diesem Punkt, überzeugt die symmetrische Verschlüsselung mit dem One-Time-Pad.
Die Abbildung \ref{sym-enc-result} zeigt ein Beispiel des Resultates.
\begin{figure}[H] 
	\centering
	\makebox[\textwidth]{\includegraphics[width=\textwidth]{img/sym-enc-result.jpg}}
	\caption[One Time Pad - Verschlüsselung]{One Time Pad - Verschlüsselung}  
	\label{sym-enc-result} 
\end{figure}

\subsection{Asymmetrische Verschlüsselung}
\subsubsection{Performance bei der asymmetrischen Verschlüsselung}
\begin{tikzpicture}
\begin{axis}[width=\textwidth, height=0.6\textwidth,
		title={Performance - Symmetrische Verschlüsselung},
		ylabel={Megapixel},
		xlabel={Sekunden}
		]
\addplot table [x=time, y=megapixel, col sep=comma] {performance-asymmetric-enc.csv};
\end{axis}
\end{tikzpicture} \\
Bei der asymmetrischen Verschlüsselung lässt die Performance zu wünschen übrig.
Bei der Verschlüsselungs eines 50 Megapixel zu verschlüsseln, brauchten wir auf unseren
Workstations durchschnittlich 337 Sekunden, also ~5.6 Minuten.
\subsubsection{Wiedererkennbarkeit des Bildes}
 Bei einigen Bildern konnte man beim verschlüsselten Bild noch Umrisse, Konturen oder grössere
 Flächen wiedererkennen.
 Somit wurden die Bilder nicht immer völlig verschleiert und teilweise waren sie wiedererkennbar.
 \begin{figure}[H] 
	\centering
	\makebox[\textwidth]{\includegraphics[width=\textwidth]{img/asym-enc-result.jpg}}
	\caption[RSA - Verschlüsselung]{RSA - Verschlüsselung}  
	\label{asym-enc-result} 
\end{figure}

\subsubsection{Problem bei der RSA Verschlüsselung} \label{problems-RSA}
\paragraph{Problem Weiss} \label{problem-white}
Wie man in der Abbildung \ref{asym-enc-result} sieht, ist der grösste Teil
der Bildes weiss.
Grund dafür ist, dass wir mit JPG Bilder von 8-Bit arbeiteten.
Mit einer Farbtiefe von 8-Bit kann man pro Farbe 255 verschiedene Werte darstellen.
Da wir nun die RGB Werte von jedem Pixel verschlüsseln, kann es also für die einzelnen
RGB Werte grössere Farbwerte als 255 geben, was eigentlich zu Farben führt,
die ausserhalb des 8-Bit Spektrums liegen.
Der Bildbetrachter von Abbildung \ref{sym-enc-result} interpretiert diese "`ungüligen"'
RGB-Werte als Weiss. 
Es gehen also nicht Informationen verloren, wie es vielleicht auf den ersten Blick schient,
sondern viele Werte liegen ausserhalb der 8-Bit Tiefe, welche dann weiss angezeigt werden.

\paragraph{Problem Performance} Die Performance mit der RSA Verschlüsselung ist schlecht.
Im Allgemeinen eignet sich die asymmetrische Verschlüsselung nicht bei grossen Dateien.
Dies zeigten auch unsere Performance Tests.

\paragraph{Problem Pixel für Pixel Verschlüsselung} 
Jedes Pixel besteht aus drei Farben: Rot, Grün und Blau.
Bei unseren Tests haben wir nur mit JPG Bildern mit einer Farbtiefe von 8-Bit gearbeitet.
Das heisst dass wir für die Farben Rot, Grün, Blau je 255 verschiedene Mögliche Werte haben können.
Unser asymmetrisches Verschlüsselungsverfahren iteriert durch jedes Pixel $P_n$ und verschlüsselt
in Jedem Pixel die Farben.
\begin{align}
Original: &P_n(R,G,B) \\
Encrypted: &P_n'(enc(R), enc(G), enc(B))
\end{align}
Durch das RSA Verschlüsselungsverfahren kann es dann aber für $enc(R|G|B)$ Werte geben die nicht mehr
im 8-Bit Bereich sind (Problem oben beschrieben).
Ein anderes Problem ist aber auch, dass der selbe RGB-Wert verschlüsselt auch immer den zum selben
verschlüsselten Wert führt. Dies erklärt auch die hellblaue Fläche im verschlüsselten Bild 
in der Abbildung \ref{asym-enc-result}

\paragraph{Keine grossen Primzahlen}
Wir wussten bereits im Vornhinein, dass wir für Matlab nur relativ kleine Primzahlen nutzen können.
Mit einer Bruteforce Attacke könnte man also unsere mit Matlab asymmetrisch verschlüsselten Bilder
in absehbarer Zeit "`entschlüsseln"'. 

 
 \subsection{Eignet sich nun Matlab für Bildverschlüsselung?} 
 Wir finden, dass sich die symmetrische Verschlüsselung mit dem One Time Pad
 gut eignet, jedoch die asymmetrische Verschlüsselung mit RSA nicht.
 
 \begin{itemize}
  \item Bei der symmetrischen Verschlüsselung gelingt es sehr einfach, den KeyStream von der
 grösse des Bildes zu erzeugen.
 \item Die XOR Operation spürt kommt auch der Performance zugute. Bei einer Bildgrösse
 von 50 Megapixel brauchten wir im Durchschnitt nicht einmal eine Zehntel Sekunde.
 \item das verschlüsselte Bild in keiner Weise wiedererkennbar.   
\end{itemize}
Die asymmetrische Verschlüsselung mit RSA eignet sich hingegen weniger, um Bilder (in Matlab) zu verschlüsseln.
 \begin{itemize}
  \item Matlab kann nicht mit grossen Primzahlen umgehen. Dies wäre jedoch ein wesentlicher Sicherheitsfaktor in RSA.
  \item Bei grossen Bildern ist die Performance sehr schlecht.
  \item Das verschlüsselte Bild ist teilweise wiedererkennbar.
\end{itemize}
 
 
 
\newpage
\section{Ahnang}
\subsection{Code}
\subsubsection{Symmetrische Verschlüsselung}
\lstinputlisting[language=Matlab]{../matlab/matlab-enc.m}
\lstinputlisting[language=Matlab]{../matlab/encData.m}

\newpage
\subsubsection{Asymmetrische Verschlüsselung}
\lstinputlisting[language=Matlab]{../matlab/matlab-enc-assymetric.m}
\lstinputlisting[language=Matlab]{../matlab/encDataRSA.m}

\end{document}