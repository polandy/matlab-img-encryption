
\input{_layout/header}
% \documentclass{article}

\begin{document}

% R�mische Nummerierung f�r Sonderseiten, wie Verzeichnisse und Anhang
\pagenumbering{Roman}

% Titelblatt
% Die Titelseite
% Im folgenden kommen ein paar Variablen, die auszuf�llen sind
% Bisher steht dort nur Musterinhalt
% Au�erdem m�ssen zei Dateien erstellt werden, Bild/Logo/Emblem des Fachgebietes
% sowie der Universit�t

\newcommand{\trtitle}{Image encryption with matlab}
\newcommand{\trtype}{Modul: Programmierem mit Matlab/Octave}
\newcommand{\trauthor}{Marco Berger\\Andy Pollari}
% \newcommand{\trstrasse}{Musterstra�e 666}
% \newcommand{\trmatrikelnummer}{0815}
\newcommand{\trort}{Bern}
\newcommand{\trbetreuer}{Prof. Marx Stampfli}
% \newcommand{\trprof}{Prof. Dr. Udo Seltsam}
% \newcommand{\trfachgebiet}{Kryptografie}
% \newcommand{\trinstitut}{gute Wissenschaft}
% \newcommand{\trfakultaet}{IV Wissenschaft}
% \newcommand{\truni}{Universit�t Musterstadt}
\newcommand{\trdate}{\today}

\thispagestyle{empty}

% Kopfzeile mit Logos.
% Eventuell die \hspace{} je nach Logogr��e anpassen
\begin{tabular}{lcr}
  \includegraphics[scale=0.8]{logo_bfh_de.jpg}% & % dein_unilogo.jpg/.eps im Verzeichnis "bilder" ablegen
 % \hspace{2cm} \truni \hspace{2cm} &
 % \includegraphics[scale=0.8]{dein_fglogo} % dein_fglogo.jpg/.eps im Verzeichnis "bilder" ablegen, Fachgebietslogo
  \\
\end{tabular}

\rule{\textwidth}{0.4pt}

\vspace{2.5cm}
\begin{center}
  \textbf{\LARGE \trtitle}
\end{center}
\vspace{2cm}

% \begin{center}
%   \textbf{\trtype} \\
%   am Fachgebiet \trfachgebiet \\
%   \trprof \\
%   Institut f�r \trinstitut \\
%   Fakult�t \trfakultaet \\
%   \truni \\[0.5cm]
%   vorgelegt von \\
%   \textbf{\trauthor}
% \end{center}

\vspace{1cm}


\begin{center}
\begin{tabular}{ll}
Betreuer: & \trbetreuer \\
\end{tabular}
\end{center}

\vfill

\begin{tabular}{l}
% \trauthor \\
% Matrikelnummer:  \trmatrikelnummer \\
% \trstrasse \\
% \trort
\end{tabular}

\rule{\textwidth}{0.4pt}

\newpage

% Verzeichnisse
% Kopfzeile links Kapitel, rechts leer
\ihead{\leftmark}
\ohead{}
\input{_layout/verzeichnisse}

% Merke mir die r�mische Seitenzahl in 'roemisch' und setzte Nummeriernung 
% auf arabisch f�r die eigentlichen Kapitel
\newpage
\newcounter{roemisch}
\setcounter{roemisch}{\value{page}}
\pagenumbering{arabic}

\chapter{Konzept}
\section{Thema}
In dieser Arbeit befassen wir uns mit der Anwendung verschiedener Verschl�sselungsalgorithmus
angewandt auf Bilder implementiert in Matlab.
Dabei untersuchen wir einerseits die Performance und ob man aus den Verschl�sselten
Bilddaten irgendwelche R�ckschl�sse ziehen kann.

\section{Fragestellung}
\begin{itemize}
  \item Wie unterschiedlich ist ein verschl�sseltes Bild mit dem Ausgangsbild?
  \item Kann man einen popul�res Bildformat wie JPEG oder PNG asynchron verschl�sseln, so dass
  das Bild immer noch korrekt interpretiert werden kann?
  \item Wie gut eignet sich Matlab f�r g�ngige Verschl�sselungsalgorithmen
\end{itemize}



\section{Skizze der erwarteten Resultate}


\section{Grundlagen}
\section{Hilfsmittel externe Daten}
\section{Mathematische Modellierung der Fragestellung}
\section{Arbeits- und Zeitplan}



% Wichtige Punkte:
% \begin{itemize}
%   \item Was will ich wissen?
%   \begin{itemize} 
%   	\item Problem schildern
%   	\item Kurze Definitionen von spezifischen Begriffen 
% 	\end{itemize}
%   \item Was habe ich f�r Daten?
%   	\begin{itemize}
%   		\item 
% 	\end{itemize}
%   \item Welche Modelle brauche ich?
%   	\begin{itemize}
%   		\item Brauche ich Suchalgorithmen?
% 		\item Tools (Toolbox bspw.Mathlab) 
% 	\end{itemize}
%   \item Was erwarte ich von den Resultaten
% 	\begin{itemize}
%   		\item 
% 	\end{itemize}
% \end{itemize}
% 
% Bestandteile des Konzepts
% \begin{itemize}
%   \item Grundlagen
%   \item Methoden
%   	\begin{itemize}
%   		\item Mathematische Methoden
% 	\end{itemize}
%   \item Vorgehen (Zeitplan, Meilensteine)
%   \item Darstellung der Resultate
% 	\begin{itemize}
% 	  \item Bilder, die etwas aussagen.
% 	\end{itemize}
%   \item Geschichte
% \end{itemize}





% Setze Numerierung wieder auf r�misch zur�ck und setzte von oben fort
% Wert ist demnach der von 'roemisch'
\newpage
\pagenumbering{Roman}
\setcounter{page}{\value{roemisch}}

% Literaturverzeichnis
\bibliography{literatur/bib}

% Appendix, falls vorhanden
\appendix
\input{_layout/anhang}

% Eidesstattliche Erkl�rung
\input{_layout/eidesstattliche}


\end{document}
